\documentclass[10pt,a4paper]{article}
\usepackage{amssymb}
\usepackage{ctex}
\usepackage{amsmath,ntheorem,lmodern,bm}
\usepackage{amsfonts}
\usepackage{geometry}
\usepackage{longtable}
\usepackage{xcolor}
\usepackage{framed}
\usepackage{tcolorbox}
\usepackage{booktabs}
\usepackage{multicol}
\usepackage{multirow}
\usepackage{mathrsfs}
\usepackage{color,framed}
%\usepackage{hyperref}
%\setCJKmainfont[BoldFont={FZHei-B01}]{FZShuSong-Z01}
%\setCJKfamilyfont{song}{FZShuSong-Z01}
%\setCJKfamilyfont{hei}{FZShuSong-Z01}
%\setCJKfamilyfont{heiti}{FZShuSong-Z01}
%\setCJKfamilyfont{heilight}{FZShuSong-Z01}
%\setCJKfamilyfont{title}{FZShuSong-Z01}
%\setCJKfamilyfont{songbold}{FZShuSong-Z01}
%\hypersetup{pdfpagemode=FullScreen}
\colorlet{shadecolor}{gray!10}
\geometry{a4paper,left=1.5cm,right=1.5cm,top=1.5cm,bottom=1.5cm}
\usepackage{ifxetex,ifluatex}
\ifxetex
  \usepackage{fontspec,xltxtra,xunicode}
  \defaultfontfeatures{Mapping=tex-text,Scale=MatchLowercase}
  \newcommand{\euro}{€}
\else
  \ifluatex
    \usepackage{fontspec}
    \defaultfontfeatures{Mapping=tex-text,Scale=MatchLowercase}
    \newcommand{\euro}{€}
  \else
    \usepackage[utf8]{inputenc}
    \usepackage{eurosym}
  \fi
\fi
\usepackage{color}
\usepackage{fancyvrb}
%\DefineShortVerb[commandchars=\\\{\}]{\|}
\DefineVerbatimEnvironment{Highlighting}{Verbatim}{commandchars=\\\{\}}
% Add ',fontsize=\small' for more characters per line
\newenvironment{Shaded}{\begin{shaded}
}{\end{shaded}}
\newcommand{\KeywordTok}[1]{\textcolor[rgb]{0.00,0.44,0.13}{\textbf{{#1}}}}
\newcommand{\DataTypeTok}[1]{\textcolor[rgb]{0.56,0.13,0.00}{{#1}}}
\newcommand{\DecValTok}[1]{\textcolor[rgb]{0.25,0.63,0.44}{{#1}}}
\newcommand{\BaseNTok}[1]{\textcolor[rgb]{0.25,0.63,0.44}{{#1}}}
\newcommand{\FloatTok}[1]{\textcolor[rgb]{0.25,0.63,0.44}{{#1}}}
\newcommand{\CharTok}[1]{\textcolor[rgb]{0.25,0.44,0.63}{{#1}}}
\newcommand{\StringTok}[1]{\textcolor[rgb]{0.25,0.44,0.63}{{#1}}}
\newcommand{\CommentTok}[1]{\textcolor[rgb]{0.38,0.63,0.69}{\textit{{#1}}}}
\newcommand{\OtherTok}[1]{\textcolor[rgb]{0.00,0.44,0.13}{{#1}}}
\newcommand{\AlertTok}[1]{\textcolor[rgb]{1.00,0.00,0.00}{\textbf{{#1}}}}
\newcommand{\FunctionTok}[1]{\textcolor[rgb]{0.02,0.16,0.49}{{#1}}}
\newcommand{\RegionMarkerTok}[1]{{#1}}
\newcommand{\ErrorTok}[1]{\textcolor[rgb]{1.00,0.00,0.00}{\textbf{{#1}}}}
\newcommand{\NormalTok}[1]{{#1}}
\ifxetex
  \usepackage[setpagesize=false, % page size defined by xetex
              unicode=false, % unicode breaks when used with xetex
              xetex,
              colorlinks=true,
              linkcolor=blue]{hyperref}
\else
  \usepackage[unicode=true,
              colorlinks=true,
              linkcolor=blue]{hyperref}
\fi
\hypersetup{breaklinks=true, pdfborder={0 0 0}}
\setlength{\parindent}{0pt}
\setlength{\parskip}{6pt plus 2pt minus 1pt}
\setlength{\emergencystretch}{3em}  % prevent overfull lines
\setcounter{secnumdepth}{0}

%\EndDefineVerbatimEnvironment{Highlighting}
\newcommand{\code}[1]{\ \tcbox[on line,
arc=3pt,outer arc=3pt,colback=pink!30,
boxsep=0pt,left=3pt,right=3pt,top=3pt,bottom=3pt,
boxrule=0pt]{\ttfamily\color{purple}{#1}}\ }

%\title{概率论第一次作业}
%\date{\today}

\begin{document}


%\section{\LaTeX 直播之七  - 数学公式入门基础配套资料}\label{latex-ux516cux5f0f}

%\section{\centerline {概率论第一次作业}}
%\section{\centerline{概率论第一次作业}}
%\section{概率论第一次作业}
{\centering\section*{概率论第一次作业}}
%\subsection{版本号: V1.6.6}\label{ux7248ux672cux53f7-v1.6.6}
%
%\textbf{编者} \\
%\subsection
\text{\rightline {黄之豪}} \\
\text{\rightline {20110980005}} \\
\text{\rightline {2020年10月16日}} \\

\subsection{作业题号}
\textbf{应坚刚教授\ 概率论(第二版)}:\\
本次作业一共10题:\\
P48 2, 3, 6, 8, 10, 12, 20, 21, 23, 24

%\textbf{编者} \\
%中山大学\ 易鹏\ \\
%华南理工大学\ 关舒文
\subsection{第2题}
解:\\
\textbf{下面的定义解读和灰色框是本题关键} \\
由定义 $$ f^{-1}(\mathscr{B}):=\{f^{-1}(B)\colon B\in\mathscr{B}	\} $$ 
	即$ f^{-1}(B)\in\mathscr{B}\iff B\in\mathscr{B} $


\subsubsection{(1)补运算}

\definecolor{shadecolor}{rgb}{0.92,0.92,0.92}
\begin{shaded}

	要证:\\
	给定条件: $ B\in\mathscr{B} $, 则$ B^{\mathrm{c}}\in\mathscr{B} $ \\
	能否得出:若$ f^{-1}(B)\in f^{-1}(\mathscr{B}) $,  
	则 $ {(f^{-1}(B))}^{\mathrm{c}} \in f^{-1}(\mathscr{B}) $

\end{shaded}

	由定义:
	\[ 
	B\in\mathscr{B} \iff f^{-1}(B)\in f^{-1}(\mathscr{B}) 
	\tag{1} 
	\] 
	\[ 
	B^{c}\in\mathscr{B} \iff f^{-1}(B^c)\in f^{-1}(\mathscr{B})
	\tag{2} 
	\] 
	
	由$ \mathscr{B} $对补运算封闭:  
	\[ 
	B\in\mathscr{B} \Rightarrow B^{c}\in\mathscr{B}  
	\tag{3} 
	\]

	综合(1)(2)(3)
	\[ 
	f^{-1}(B)\in f^{-1}(\mathscr{B})  \Rightarrow f^{-1}(B^c)\in f^{-1}(\mathscr{B})
	\]
	由逆运算与补集交换性(课本P40定理2.3.1(2)): $  f^{-1}(B^c)={(f^{-1}(B))}^c  $ , 可得:
	\[ 
	f^{-1}(B)\in f^{-1}(\mathscr{B})  \Rightarrow {(f^{-1}(B))}^c \in f^{-1}(\mathscr{B})
	\]

\subsubsection{(2)并运算}

\definecolor{shadecolor}{rgb}{0.92,0.92,0.92}
\begin{shaded}
	
	要证:\\
	给定条件: $ A,B\in\mathscr{B} $, 则$ A \cup B \in\mathscr{B} $ \\
	能否得出:若$ f^{-1}(A),f^{-1}(B)\in f^{-1}(\mathscr{B}) $,  
	则 $ f^{-1}(A) \cup f^{-1}(B) \in f^{-1}(\mathscr{B}) $
	
\end{shaded}

由定义:
\[ 
A,B\in\mathscr{B} \iff f^{-1}(A),f^{-1}(B)\in f^{-1}(\mathscr{B}) 
\tag{1} 
\] 
\[ 
A \cup B \in\mathscr{B} \iff f^{-1}(A \cup B) \in f^{-1}(\mathscr{B})
\tag{2} 
\] 

由$ \mathscr{B} $对并运算封闭:  
\[ 
A,B\in\mathscr{B} \Rightarrow A \cup B \in\mathscr{B}
\tag{3} 
\]

综合(1)(2)(3)
\[ 
f^{-1}(A),f^{-1}(B)\in f^{-1}(\mathscr{B})  \Rightarrow f^{-1}(A \cup B) \in f^{-1}(\mathscr{B})
\]
由逆运算与并集交换性(课本P40定理2.3.1(3)): $  f^{-1}(A \cup B)=f^{-1}(A) \cup f^{-1}(B)  $ , 可得:
\[ 
f^{-1}(A),f^{-1}(B)\in f^{-1}(\mathscr{B})  \Rightarrow f^{-1}(A) \cup f^{-1}(B) \in f^{-1}(\mathscr{B})
\]

\subsubsection{(3)交运算}

\definecolor{shadecolor}{rgb}{0.92,0.92,0.92}
\begin{shaded}
	
	要证:\\
	给定条件: $ A,B\in\mathscr{B} $, 则$ A \cap B \in\mathscr{B} $ \\
	能否得出:若$ f^{-1}(A),f^{-1}(B)\in f^{-1}(\mathscr{B}) $,  
	则 $ f^{-1}(A) \cap f^{-1}(B) \in f^{-1}(\mathscr{B}) $
	
\end{shaded}

由定义:
\[ 
A,B\in\mathscr{B} \iff f^{-1}(A),f^{-1}(B)\in f^{-1}(\mathscr{B}) 
\tag{1}
\] 
\[ 
A \cap B \in\mathscr{B} \iff f^{-1}(A \cap B) \in f^{-1}(\mathscr{B})
\tag{2} 
\] 

由$ \mathscr{B} $对交运算封闭:  
\[ 
A,B\in\mathscr{B} \Rightarrow A \cap B \in\mathscr{B}
\tag{3} 
\]

综合(1)(2)(3)
\[ 
f^{-1}(A),f^{-1}(B)\in f^{-1}(\mathscr{B})  \Rightarrow f^{-1}(A \cap B) \in f^{-1}(\mathscr{B})
\]
由逆运算与交集交换性(课本P40定理2.3.1(3)): $  f^{-1}(A \cap B)=f^{-1}(A) \cap f^{-1}(B)  $ , 可得:
\[ 
f^{-1}(A),f^{-1}(B)\in f^{-1}(\mathscr{B})  \Rightarrow f^{-1}(A) \cap f^{-1}(B) \in f^{-1}(\mathscr{B})
\]


\subsection{第3题}
解:\\
\textbf{与2解题几乎一致} \\
由定义 $$ \mathscr{B}:=\{B \subset Y \colon  f^{-1}(B) \in \mathscr{A}	\} $$ 
即$ B\in\mathscr{B}\iff f^{-1}(B) \in \mathscr{A} $


\subsubsection{(1)补运算}

\definecolor{shadecolor}{rgb}{0.92,0.92,0.92}
\begin{shaded}
	
	要证:\\
	给定条件: $ f^{-1}(B)\in\mathscr{A} $, 则$ {(f^{-1}(B))}^c\in\mathscr{A} $ \\
	能否得出:若$ B\in \mathscr{B} $,  
	则 $ B^c \in \mathscr{B} $
	
\end{shaded}

由定义:
\[ 
f^{-1}(B)\in \mathscr{A} \iff  B\in\mathscr{B} 
\tag{1} 
\] 
\[ 
f^{-1}(B^c)\in \mathscr{A} \iff B^{c}\in\mathscr{B}
\tag{2} 
\] 

由$ \mathscr{A} $对补运算封闭:  
\[ 
f^{-1}(B)\in\mathscr{A} \Rightarrow {(f^{-1}(B))}^c\in\mathscr{A}  
\tag{3} 
\]

综合(1)(2)(3)
\[ 
B\in\mathscr{B} \iff f^{-1}(B)\in \mathscr{A}  \Rightarrow {(f^{-1}(B))}^c\in\mathscr{A}
\]
由逆运算与补集交换性(课本P40定理2.3.1(2)): $  f^{-1}(B^c)={(f^{-1}(B))}^c  $ , 再由(2)可得:
\[ 
B\in\mathscr{B}  \Rightarrow  B^{c}\in\mathscr{B}
\]

\subsubsection{(2)并运算}

\definecolor{shadecolor}{rgb}{0.92,0.92,0.92}
\begin{shaded}
	要证:\\
	给定条件: $ f^{-1}(A),f^{-1}(B)\in\mathscr{A} $, 则$ f^{-1}(A) \cup f^{-1}(B) \in\mathscr{A} $ \\
	能否得出:若$ A,B\in \mathscr{B} $,  
	则 $ A \cup B \in \mathscr{B} $
	
	
\end{shaded}

由定义:
\[ 
f^{-1}(A),f^{-1}(B)\in\mathscr{A} \iff A,B\in \mathscr{B}
\tag{1} 
\] 


由$ \mathscr{A} $对并运算封闭:  
\[ 
f^{-1}(A),f^{-1}(B)\in\mathscr{A} \Rightarrow f^{-1}(A) \cup f^{-1}(B) \in\mathscr{A}
\tag{2} 
\]
由逆运算与并集交换性(课本P40定理2.3.1(3)): $  f^{-1}(A \cup B)=f^{-1}(A) \cup f^{-1}(B)  $ , 结合(1)(2)可得:
\[ 
A,B\in \mathscr{B} \iff f^{-1}(A),f^{-1}(B)\in\mathscr{A} \Rightarrow f^{-1}(A) \cup f^{-1}(B)=f^{-1}(A \cup B) \in\mathscr{A}
\]

又由定义:
\[ 
A \cup B \in\mathscr{B} \iff f^{-1}(A \cup B) \in f^{-1}(\mathscr{B})
\]
得: 
\[ 
A,B\in \mathscr{B}  \Rightarrow A \cup B \in \mathscr{B}
\]

\subsubsection{(3)交运算}

\definecolor{shadecolor}{rgb}{0.92,0.92,0.92}
\begin{shaded}
	要证:\\
	给定条件: $ f^{-1}(A),f^{-1}(B)\in\mathscr{A} $, 则$ f^{-1}(A) \cap f^{-1}(B) \in\mathscr{A} $ \\
	能否得出:若$ A,B\in \mathscr{B} $,  
	则 $ A \cap B \in \mathscr{B} $
	
	
\end{shaded}

由定义:
\[ 
f^{-1}(A),f^{-1}(B)\in\mathscr{A} \iff A,B\in \mathscr{B}
\tag{1} 
\] 


由$ \mathscr{A} $对并运算封闭:  
\[ 
f^{-1}(A),f^{-1}(B)\in\mathscr{A} \Rightarrow f^{-1}(A) \cap f^{-1}(B) \in\mathscr{A}
\tag{2} 
\]
由逆运算与交集交换性(课本P40定理2.3.1(3)): $  f^{-1}(A \cap B)=f^{-1}(A) \cap f^{-1}(B)  $ , 结合(1)(2)可得:
\[ 
A,B\in \mathscr{B} \iff f^{-1}(A),f^{-1}(B)\in\mathscr{A} \Rightarrow f^{-1}(A) \cap f^{-1}(B)=f^{-1}(A \cap B) \in\mathscr{A}
\]

又由定义:
\[ 
A \cap B \in\mathscr{B} \iff f^{-1}(A \cap B) \in f^{-1}(\mathscr{B})
\]
得: 
\[ 
A,B\in \mathscr{B}  \Rightarrow A \cap B \in \mathscr{B}
\]

\subsection{第6题}
解:\\
设: \\
\centerline{
$ {\forall n}  $, $  \mathscr{F}_{1},\mathscr{F}_{2} \cdots \mathscr{F}_{n} $ 是$ \sigma $域}
(1)\\
显然:
\[ \varnothing \in \mathscr{F}_{1} \cap \mathscr{F}_{2} \cdots \cap \mathscr{F}_{n} 
\tag{a}
\]
由$ \sigma $域定义:
\[ \Omega \in \mathscr{F}_{1}, \Omega \in \mathscr{F}_{2} \cdots \Omega \in \mathscr{F}_{n} 
\]
得:
\[ \Omega \in \mathscr{F}_{1} \cap \mathscr{F}_{2} \cdots \cap \mathscr{F}_{n} 
\tag{b}
\]
(2)\\
设$ \forall A$ :
\[ \forall A \in \mathscr{F}_{1} \cap \mathscr{F}_{2} \cdots \cap \mathscr{F}_{n} \]
得:
\[ A \in \forall \mathscr{F}_{i}, i=1,2,\cdots,n \]
由$ \sigma $域定义可得:
 \[ A^{\mathrm{c}} \in \forall \mathscr{F}_{i}, i=1,2,\cdots,n \] 
即
\[ A^{\mathrm{c}} \in \mathscr{F}_{1} \cap \mathscr{F}_{2} \cdots \cap \mathscr{F}_{n}
\tag{c}
\]  
(3)\\
设:
\[ A_{k} \in \mathscr{F}_{1} \cap \mathscr{F}_{2} \cdots \cap \mathscr{F}_{n}, k\geq 1
\] 
则:
\[ \bigcup_{k} A_{k} \in \mathscr{F}_{1} , \bigcup_{k} A_{k} \in \mathscr{F}_{2}, \cdots \bigcup_{k} A_{k} \in \mathscr{F}_{n}, k\geq 1
\] 
可得:
\[ \bigcup_{k} A_{k} \in \mathscr{F}_{1} \cap \mathscr{F}_{2} \cdots \cap \mathscr{F}_{n}, k\geq 1
\tag{d}
\] 
由(a)(b)(c)(d)得证

\subsection{第8题}
解:\\ 
\begin{shaded}
	
	其实$(b) \Rightarrow (a)$也又看到证明,$(c) \Rightarrow (a)$不一定对
	
\end{shaded}
使用$ (a) \Rightarrow (b) \Rightarrow (c) \Rightarrow (a) $思路\\
$ (a) \Rightarrow (b)$: \\
显然,将n+1之后的集合都设为空集,便从可列可加性$\Rightarrow$有限可加\\
现证下连续:\\
设$\{ S_{n} \}$是$ \mathscr{F} $中一个单调不减的集序列,那么:
\[ \bigcup_{i=1}^{\infty} S_{i}= \lim\limits_{n\to+\infty} S_{n} \]
若定义$ S_{0}=\varnothing $,则
\[ \bigcup_{i=1}^{\infty} S_{i}= \sum_{i=1}^{\infty} (S_{i}-S_{i-1}) \]
这里的$ (S_{i} - S_{i-1}) , i=1,2,\cdots$由于$S_{i}$单调性,两两互不相容,因此由可列可加性得
\[ \mathbb{P}(\bigcup_{i=1}^{\infty} S_{i}) = \sum_{i=1}^{\infty} \mathbb{P}(S_{i}-S_{i-1}) = \lim\limits_{n\to+\infty} \sum_{i=1}^{n} \mathbb{P}(S_{i}-S_{i-1}) \]
又由
\[ \sum_{i=1}^{n} \mathbb{P}(S_{i}-S_{i-1})= \mathbb{P}(\sum_{i=1}^{n} (S_{i}-S_{i-1})) = \mathbb{P}(S_{n})\]
因此
\[  \mathbb{P}(\lim\limits_{n\to+\infty} S_{n}) = \lim\limits_{n\to+\infty} \mathbb{P}(S_{n}) \]
下连续性得证\\


$ (b) \Rightarrow (c)$: \\
下连续性$\Rightarrow$次可列可加:\\
利用一般加法定理(容斥原理)
\[ \mathbb{P}(\bigcup_{k=1}^{n} A_{k}) = \sum_{k=1}^{n} \mathbb{P}(A_{k})-\sum_{1 \leq i<j \leq n} \mathbb{P}((A_{i} \cap A_{j}) + \sum_{1 \leq i<j<k \leq n} \mathbb{P}((A_{i} \cap A_{j} \cap A_{k}) - \cdots + (-1)^{n-1}\mathbb{P}(A_{1} \cap \cdots \cap A_{n}) \]
对n作数学归纳可得,对$\forall n \geq 1$有
\[ \mathbb{P}(\bigcup_{k=1}^{n} A_{k}) \leq \sum_{k=1}^{n} \mathbb{P}(A_{k}) \]
令$n\to +\infty$,根据下连续性可得
\[ \mathbb{P}(\bigcup_{k=1}^{\infty} A_{k}) \leq \sum_{k=1}^{\infty} \mathbb{P}(A_{k}) \]


$ (c) \Rightarrow (a)$: \\
由条件可得$\mathbb{P}$首先是外测度,设$ \{A_{i}\} $是一列互不相交的可测集,由$\left( \Omega,\mathscr{F}\right)$是可测空间可得,$\bigcup_{i=1}^{\infty} A_{i}, A_{i} \in \mathscr{F} $是可测集,根据外测度性质,对$\forall T$

%\begin{equation*}
%	\begin{split}
%	\mathbb{P}(T)  
%	& = \mathbb{P}\left[ T \cap \left( \bigcup_{i=1}^{n} A_{i} \right) \right] + \mathbb{P}\left[ T \cap \left( \bigcup_{i=1}^{n} A_{i} \right)^\mathrm{c} \right] \\
%	& \geq \mathbb{P}\left[ T \cap \left( \bigcup_{i=1}^{n} A_{i} \right) \right] + \mathbb{P}\left[ T \cap \left( \bigcup_{i=1}^{\infty} A_{i} \right)^\mathrm{c} \right]  \\
%	& = \sum_{i=1}^{n} \mathbb{P}(T \cap A_{i}) + \mathbb{P}\left[ T \cap \left( \bigcup_{i=1}^{\infty} A_{i} \right)^\mathrm{c} \right]	\\
%	\end{split}
%\end{equation*}


\begin{align}
	\mathbb{P}(T)  
	& = \mathbb{P}\left[ T \cap \left( \bigcup_{i=1}^{n} A_{i} \right) \right] + \mathbb{P}\left[ T \cap \left( \bigcup_{i=1}^{n} A_{i} \right)^\mathrm{c} \right] \notag \\
	& \geq \mathbb{P}\left[ T \cap \left( \bigcup_{i=1}^{n} A_{i} \right) \right] + \mathbb{P}\left[ T \cap \left( \bigcup_{i=1}^{\infty} A_{i} \right)^\mathrm{c} \right] \notag \\
	& = \sum_{i=1}^{n} \mathbb{P}(T \cap A_{i}) + \mathbb{P}\left[ T \cap \left( \bigcup_{i=1}^{\infty} A_{i} \right)^\mathrm{c} \right]	\tag{*}
\end{align}
其中(*)用到了有限可加性,令$n \to \infty$得
\[ \mathbb{P}(T) \geq \sum_{i=1}^{\infty} \mathbb{P}(T \cap A_{i}) + \mathbb{P}\left[ T \cap \left( \bigcup_{i=1}^{\infty} A_{i} \right)^\mathrm{c} \right] \tag{**}\]
在(**)中,令$T=\bigcup_{i=1}^{\infty} A_{i}$,这时由于$ (\bigcup_{i=1}^{\infty} A_{i} \cap A_{i}=A_{i}) $,便得
\[ 
\mathbb{P}(\bigcup_{i=1}^{\infty} A_{i}) \geq \sum_{i=1}^{\infty} \mathbb{P}(A_{i})
\]
另一方面由次可列可加性
\[ \mathbb{P}(\bigcup_{k=1}^{\infty} A_{k}) \leq \sum_{k=1}^{\infty} \mathbb{P}(A_{k}) \]
最终可得可列可加性

\subsection{第10题}
解:\\ 
由$ A \cap B \subset B  $以及概率单调性,得出   $ \mathbb{P}(A \cap B) \leq 1/3 $ \\
再由$ 1 \geq \mathbb{P}(A \cup B)=\mathbb{P}(A) + \mathbb{P}(B) - \mathbb{P}(A \cap B) $, 得$\mathbb{P}(A \cup B) \geq 1/12$


\subsection{第12题}
解:\\ 
用数学归纳法证明:\\
当$n=2$时, $ \mathbb{P}(A_{1} \cup A_{2})=\mathbb{P}(A_{1}) + \mathbb{P}(A_{2}) - \mathbb{P}((A_{1} \cap A_{2})  $ , 显然成立:$ \mathbb{P}(A_{1} \cup A_{2}) \geq \mathbb{P}(A_{1}) + \mathbb{P}(A_{2}) - \mathbb{P}((A_{1} \cap A_{2})  $ \\
设当$n=k-1$时,\[ \mathbb{P}(\bigcup_{i=1}^{k-1} A_{i}) \geq \sum_{i=1}^{k-1} \mathbb{P}(A_{i})-\sum_{i<j \leq k-1} \mathbb{P}((A_{i} \cap A_{j}) \]
当$n=k$时,
\[ 
\mathbb{P}(\bigcup_{i=1}^{k} A_{i})=\mathbb{P}((\bigcup_{i=1}^{k-1} A_{i}) \cup A_{k})=\mathbb{P}(\bigcup_{i=1}^{k-1} A_{i})+\mathbb{P}(A_{k})-\mathbb{P}(\bigcup_{i=1}^{k-1} A_{i} \cap A_{k})
\tag{1}
\]
其中,利用$n=k-1$假设,
\[ \mathbb{P}(\bigcup_{i=1}^{k-1} A_{i}) \geq \sum_{i=1}^{k-1} \mathbb{P}(A_{i})-\sum_{i<j \leq k-1} \mathbb{P}((A_{i} \cap A_{j}) 
\tag{2}
\]
使用概率分配率和次可加性
\[ \mathbb{P}(\bigcup_{i=1}^{k-1} A_{i} \cap A_{k}) = \mathbb{P}(\bigcup_{i=1}^{k-1} (A_{i} \cap A_{k}) \leq \sum_{i=1}^{k-1}\mathbb{P}(A_{i} \cap A_{k}) 
\tag{3}
\] 
将(2)(3)代入(1)便得证
\[ \mathbb{P}(\bigcup_{i=1}^{k} A_{i}) \geq \sum_{i=1}^{k} \mathbb{P}(A_{i})-\sum_{i<j \leq k} \mathbb{P}((A_{i} \cap A_{j}) \]


\subsection{第20题}
解:\\ 
\begin{shaded}
	
	这题前半小题有点麻烦,要证明连续分布函数F可测(可测函数为啥是随机变量李贤平P158具体不清楚)\\
	PS:好像记忆中甚至可以不用连续这一条件,可测还是比连续要弱一点的
	
\end{shaded}
(1)证明$F(x)$是可测函数(一元Borel函数),即$F(\xi)$是随机变量\\
对$ \forall a$, 记 $E = \{x \in \mathbb{R} \colon F(x)>a \}  $ \\
现在证明E是开集:$ \forall x_{0} \in E$,则$F(x_{0})>a$ 由于$F(x)$在$x_{0}$处连续,$ \exists \delta>0$,使得$x \in U(x_{0}, \delta)$时,$F(x)>a$,即$x_{0}$是内点,E是开集,当$a \geq 1$或$a \leq 0$时,E分别是$\varnothing$和$\mathbb{R}$也都为开集,开集是可测集,因此$F(x)$是可测函数,故$F(\xi)$是随机变量。\\

(2)证明$[0,1]$均匀分布,连续单调函数有逆函数\\
$Y=F(\xi)$分布:$\mathbb{P}(Y \leq y)=\mathbb{P}(F(\xi) \leq y)=\mathbb{P}(\xi \leq F^{-1}(y))=F(F^{-1}(y))=y$,因此是$[0,1]$均匀分布


\subsection{第21题}
解:\\ 
\begin{shaded}
	
	这题也麻烦,广义逆(教材P97)
	
\end{shaded}

(1)至少存在中点\\
广义逆定义:
\[ F^{-1}(x) \triangleq inf \{ y \colon F(y) \geq x \} \]
反证法,取$m=F^{-1}(\frac{1}{2})$,若$ \mathbb{P}(\xi < m)>\frac{1}{2} $,由$F^{-1}$左连续性,$\exists m_{0}$使 $ \mathbb{P}(\xi \leq m_{0})\geq \frac{1}{2} $,与m是下确界矛盾


(2)中点集合闭区间\\
首先不会出现中点集合为多段区间情况(否则与概率单调不减矛盾),反证法,不妨设中点区间为$(a,b]$,则\\
\[ \exists N, n>N, \mathbb{P}(\xi < a+\frac{1}{n})\leq \frac{1}{2}\leq \mathbb{P}(\xi \leq a+\frac{1}{n}) \]
上式两边取极限$n \to +\infty $,${\xi < a+\frac{1}{n}}$为单调递减集合,由概率上连续性得
\[ \mathbb{P}(\xi < a)\leq \frac{1}{2}\leq \mathbb{P}(\xi \leq a) \]
此时a符合中点定义,即与中点集合为(a,b]矛盾


\subsection{第23题}
解:\\ 
\begin{shaded}
	
	思路是利用概率减法性质把两个概率相减变为一个
	
\end{shaded}

不妨设 $\mathbb{P}(\xi \in I) \geq \mathbb{P}(\eta \in I) $,则
\begin{equation*}
	\begin{split}
		  \mathbb{P}(\xi \in I)-\mathbb{P}(\eta \in I)  
		 & \leq \mathbb{P}(\xi \in I)-\mathbb{P}(\eta \in I, \xi \in I) \\
		 & =\mathbb{P}(\eta \in I, \xi \notin I)  \\
		 & \leq \mathbb{P}(\eta \neq \xi )	 
	\end{split}
\end{equation*}

\subsection{第24题}
解:\\ 

(1)联合连续$\Rightarrow$边缘连续:\\
联合分布连续$\Rightarrow$每个分量连续,则显然$F(x, +\infty)$和$F(+\infty,y)$连续 \\

(2)边缘连续$\Rightarrow$联合连续\\
边缘连续,则$\forall x_{0}, \forall \varepsilon, \exists \delta>0$,使得当$\left| x-x_{0} \right|<\delta$时,有$\left| F_{\xi}(x)-F_{\xi}(x_{0}) \right|<\varepsilon$和$\left| F_{\eta}(x)-F_{\eta}(x_{0}) \right|<\varepsilon$ \\
$\Rightarrow$对于$\forall x_{0},y_{0}$,满足$\left| x-x_{0} \right|<\delta, \left| y-y_{0} \right|<\delta$时,有\\

\begin{equation*}
	\begin{split}
	\left| F(x,y)-F(x_{0},y_{0}) \right|  
	& \leq \left| F(x,y)-F(x_{0},y) \right| + \left| F(x_{0},y)-F(x_{0},y_{0}) \right| \\
	& =\mathbb{P}(min(x,x_{0})<\xi \leq max(x,x_{0}), \eta \leq y) +  \mathbb{P}(min(y,y_{0})<\eta \leq max(y,y_{0}), \xi \leq x_{0}) \\
	& \leq \mathbb{P}(min(x,x_{0})<\xi \leq max(x,x_{0}), \eta \leq +\infty) +  \mathbb{P}(min(y,y_{0})<\eta \leq max(y,y_{0}), \xi \leq +\infty)	\\
	& = \left| F_{\xi}(x)-F_{\xi}(x_{0}) \right| + \left| F_{\eta}(x)-F_{\eta}(x_{0}) \right| < 2\varepsilon
	\end{split}
\end{equation*}

\end{document} 