\documentclass[10pt,a4paper]{article}
\usepackage{amssymb}
\usepackage{ctex}
\usepackage{amsmath,ntheorem,lmodern,bm}
\usepackage{amsfonts}
\usepackage{geometry}
\usepackage{longtable}
\usepackage{xcolor}
\usepackage{framed}
\usepackage{tcolorbox}
\usepackage{booktabs}
\usepackage{multicol}
\usepackage{multirow}
\usepackage{mathrsfs}
\usepackage{color,framed}
%\usepackage{hyperref}
%\setCJKmainfont[BoldFont={FZHei-B01}]{FZShuSong-Z01}
%\setCJKfamilyfont{song}{FZShuSong-Z01}
%\setCJKfamilyfont{hei}{FZShuSong-Z01}
%\setCJKfamilyfont{heiti}{FZShuSong-Z01}
%\setCJKfamilyfont{heilight}{FZShuSong-Z01}
%\setCJKfamilyfont{title}{FZShuSong-Z01}
%\setCJKfamilyfont{songbold}{FZShuSong-Z01}
%\hypersetup{pdfpagemode=FullScreen}
\colorlet{shadecolor}{gray!10}
\geometry{a4paper,left=1.5cm,right=1.5cm,top=1.5cm,bottom=1.5cm}
\usepackage{ifxetex,ifluatex}
\ifxetex
  \usepackage{fontspec,xltxtra,xunicode}
  \defaultfontfeatures{Mapping=tex-text,Scale=MatchLowercase}
  \newcommand{\euro}{€}
\else
  \ifluatex
    \usepackage{fontspec}
    \defaultfontfeatures{Mapping=tex-text,Scale=MatchLowercase}
    \newcommand{\euro}{€}
  \else
    \usepackage[utf8]{inputenc}
    \usepackage{eurosym}
  \fi
\fi
\usepackage{color}
\usepackage{fancyvrb}
%\DefineShortVerb[commandchars=\\\{\}]{\|}
\DefineVerbatimEnvironment{Highlighting}{Verbatim}{commandchars=\\\{\}}
% Add ',fontsize=\small' for more characters per line
\newenvironment{Shaded}{\begin{shaded}
}{\end{shaded}}
\newcommand{\KeywordTok}[1]{\textcolor[rgb]{0.00,0.44,0.13}{\textbf{{#1}}}}
\newcommand{\DataTypeTok}[1]{\textcolor[rgb]{0.56,0.13,0.00}{{#1}}}
\newcommand{\DecValTok}[1]{\textcolor[rgb]{0.25,0.63,0.44}{{#1}}}
\newcommand{\BaseNTok}[1]{\textcolor[rgb]{0.25,0.63,0.44}{{#1}}}
\newcommand{\FloatTok}[1]{\textcolor[rgb]{0.25,0.63,0.44}{{#1}}}
\newcommand{\CharTok}[1]{\textcolor[rgb]{0.25,0.44,0.63}{{#1}}}
\newcommand{\StringTok}[1]{\textcolor[rgb]{0.25,0.44,0.63}{{#1}}}
\newcommand{\CommentTok}[1]{\textcolor[rgb]{0.38,0.63,0.69}{\textit{{#1}}}}
\newcommand{\OtherTok}[1]{\textcolor[rgb]{0.00,0.44,0.13}{{#1}}}
\newcommand{\AlertTok}[1]{\textcolor[rgb]{1.00,0.00,0.00}{\textbf{{#1}}}}
\newcommand{\FunctionTok}[1]{\textcolor[rgb]{0.02,0.16,0.49}{{#1}}}
\newcommand{\RegionMarkerTok}[1]{{#1}}
\newcommand{\ErrorTok}[1]{\textcolor[rgb]{1.00,0.00,0.00}{\textbf{{#1}}}}
\newcommand{\NormalTok}[1]{{#1}}
\ifxetex
  \usepackage[setpagesize=false, % page size defined by xetex
              unicode=false, % unicode breaks when used with xetex
              xetex,
              colorlinks=true,
              linkcolor=blue]{hyperref}
\else
  \usepackage[unicode=true,
              colorlinks=true,
              linkcolor=blue]{hyperref}
\fi
\hypersetup{breaklinks=true, pdfborder={0 0 0}}
\setlength{\parindent}{0pt}
\setlength{\parskip}{6pt plus 2pt minus 1pt}
\setlength{\emergencystretch}{3em}  % prevent overfull lines
\setcounter{secnumdepth}{0}

%\EndDefineVerbatimEnvironment{Highlighting}
\newcommand{\code}[1]{\ \tcbox[on line,
arc=3pt,outer arc=3pt,colback=pink!30,
boxsep=0pt,left=3pt,right=3pt,top=3pt,bottom=3pt,
boxrule=0pt]{\ttfamily\color{purple}{#1}}\ }

%\title{概率论第一次作业}
%\date{\today}

\begin{document}


%\section{\LaTeX 直播之七  - 数学公式入门基础配套资料}\label{latex-ux516cux5f0f}

%\section{\centerline {概率论第一次作业}}
%\section{\centerline{概率论第一次作业}}
%\section{概率论第一次作业}
{\centering\section*{概率论第二次作业}}
%{\centering\section*{Advanced Statistical Learning}}
%\subsection{版本号: V1.6.6}\label{ux7248ux672cux53f7-v1.6.6}
%
%\textbf{编者} \\
%\subsection
\begin{flushright}
	\Large {\text{黄之豪}} \\
	\Large {\text{20110980005}} \\
	\Large {\text{Oct 28th, 2020}} \\
\end{flushright}


%\textbf{编者} \\
%中山大学\ 易鹏\ \\
%华南理工大学\ 关舒文
\subsection{P66 18}
设 $ A=\{ \mbox{第k张是前k张中最大的} \} $,\ $ B=\{ \mbox{第k张是m} \} $, 显然B发生一定能推出A发生 \\
$ \mathbb{P}(A): $ 从m中随机抽k个,第k个位置放置最大数,前k-1个位置随意放置(排列)。$ \mathbb{P}(B): $ 从m-1中随机抽k-1个,第k个位置放置m,前k-1个位置随意放置(排列)。
$$ \mathbb{P}(B|A)=\frac{\mathbb{P}(AB)}{\mathbb{P}(A)}=\frac{\mathbb{P}(B)}{\mathbb{P}(A)}=\frac{\frac{{m-1 \choose k-1} (k-1)!}{{m \choose k} k!}}{\frac{{m \choose k} (k-1)!}{{m \choose k} k!}}=\frac{k}{m} 
$$ 




\subsection{P66 20}
两独立Possion分布随机变量之和也服从Possion分布且参数为两者之和
\begin{align}
	\mathbb{P}(\xi_1+\xi_2=n)
	& = \sum_{k=0}^{n}\mathbb{P}(\xi_1=k)\mathbb{P}(\xi_1+\xi_2=n | \xi_1=k) \notag \\
	& = \sum_{k=0}^{n}\mathbb{P}(\xi_1=k)\mathbb{P}(\xi_2=n-k) \notag \\
	& = \sum_{k=0}^{n} \frac{\lambda_1^k}{k!}e^{-\lambda_1} \frac{\lambda_2^{n-k}}{(n-k)!}e^{-\lambda_2} \notag \\
	& = e^{-(\lambda_1+\lambda_2)}\sum_{k=0}^{n} \frac{\lambda_1^k \lambda_2^{n-k}}{k!(n-k)!} \notag \\
	& = e^{-(\lambda_1+\lambda_2)}\sum_{k=0}^{n} \frac{{n \choose k} \lambda_1^k \lambda_2^{n-k}}{n!} \notag \\
	& = e^{-(\lambda_1+\lambda_2)} \frac{(\lambda_1+\lambda_2)^n}{n!} \notag 
\end{align}

$$ \mathbb{P}(\xi_1=k | \xi_1+\xi_2=n)=\frac{\mathbb{P}(\xi_1=k , \xi_1+\xi_2=n)}{\mathbb{P}(\xi_1+\xi_2=n)}=\frac{\mathbb{P}(\xi_1=k)\mathbb{P}(\xi_1+\xi_2=n)}{\mathbb{P}(\xi_1+\xi_2=n)}=\frac{{n \choose k} \lambda_1^k \lambda_2^{n-k}}{(\lambda_1 + \lambda_2)^n} $$



\subsection{P67 24}
起始状态(状态A):白白白  黑黑黑 \\
第二步状态(状态B): 白白黑 黑黑白 \\

从上面可以发现,状态B除非同时抽到唯一的黑和唯一的白做交换会回到状态A(概率值为1/3 * 1/3 = 1/9),否则其余情况都是回到状态B(最终只关心颜色一致,并不关心一致的颜色在哪个盒子)

转移概率矩阵T为:
\[ 
\begin{array}{ccc}
  & A & B\\\\
A & 0 & 1\\\\
B & 1/9 & 8/9\\\\
\end{array} 
\] 
n次之后仍然颜色相同,也即n次之后回到状态A的概率,这一概率值其实是转移矩阵n次幂之后位于(1,1)位置(矩阵的左上角)的数\\
$ {\begin{pmatrix} 0&1\\ \frac{1}{9}& \frac{8}{9} \end{pmatrix}}^n $ 求解则可通过特征分解方法\\
$ {\begin{pmatrix} 0&1\\ \frac{1}{9}& \frac{8}{9} \end{pmatrix}} $ 的特征值为 $\lambda_1=1, \lambda_2=-\frac{1}{9}$ , 对应特征向量为   $ \xi_1={\left(\begin{array}{cc} 1 & 1  \end{array} \right)}^T $,$ \xi_2={\left(\begin{array}{cc} -9 & 1  \end{array} \right)}^T $

\[ T^n=
\left(
\begin{array}{cc}
1 & -9 \\
1 & 1
\end{array}
\right)
{\begin{pmatrix} 1&0\\ 0& -\frac{1}{9} \end{pmatrix}}^n
\left(
\begin{array}{cc}
\frac{1}{10} & \frac{9}{10} \\
-\frac{1}{10} & \frac{1}{10}
\end{array}
\right) = \left(
\begin{array}{cc}
\frac{1}{10}+\frac{9}{10}(-\frac{1}{10})^n & * \\
* & *
\end{array}
\right)
 \]
因此n次之后颜色仍然相同概率为 $ \frac{1}{10}+\frac{9}{10}(-\frac{1}{10})^n $


\subsection{P67 26}
掷两个骰子共有6*6=36种可能,和为7的概率为6/36;和为11的概率2/36;和为2,3,12的概率1/36,2/36,1/36;和为6,8的概率都为5/36;和为5,9的概率都为4/36;和为4,10的概率都为3/36 \\


\begin{align}
\mathbb{P}(win)
& = \frac{2}{9} \tag{1} \\
& + 2 \times \frac{5}{36}(\frac{5}{36}+\frac{25}{36}\times\frac{5}{36}+\frac{25}{36}\times\frac{25}{36}\times\frac{5}{36}+ \cdots) \tag{2} \\
& + 2 \times \frac{4}{36}(\frac{4}{36}+\frac{26}{36}\times\frac{4}{36}+\frac{26}{36}\times\frac{26}{36}\times\frac{4}{36}+ \cdots) \tag{3} \\
& + 2 \times \frac{3}{36}(\frac{3}{36}+\frac{27}{36}\times\frac{3}{36}+\frac{27}{36}\times\frac{27}{36}\times\frac{3}{36}+ \cdots) \tag{4} \\
& = \frac{2}{9}+\frac{10}{36}\times\frac{5}{36}\times\frac{1}{1-\frac{25}{36}} + \frac{8}{36}\times\frac{4}{36}\times\frac{1}{1-\frac{26}{36}}+\frac{6}{36}\times\frac{3}{36}\times\frac{1}{1-\frac{27}{36}} \notag \\
& = \frac{244}{495} \notag
\end{align}

分析:上式(1)代表第一次就赢(即掷到7,11)概率为8/36=2/9,上式(2)代表第一次和为6或8,进入下一次投掷,比如第一次和为6,第二次投掷5/36的概率赢,1-5/36-6/36=25/36的概率进入第三轮投掷,一直持续下去,同理,上式(3)代表第一次和为5或9,上式(4)代表第一次和为4或10,最终得到赌徒获胜概率为244/495。




\end{document} 